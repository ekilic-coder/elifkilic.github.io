\documentclass[9pt,twocolumn,twoside]{pnas-new}
% Use the lineno option to display guide line numbers if required.

\templatetype{pnasresearcharticle} % Choose template

\begin{document}

\title{A Heat Index for Laboring Populations}

\author[a,1]{Elif Kilic}
\author[a,b]{Ashok Gadgil}

\affil[a]{University of California, Berkeley}
\affil[b]{Lawrence Berkeley National Laboratory}

\leadauthor{Kilic}

\significancestatement{Global warming is pushing millions beyond the human climate niche, particularly outdoor workers in tropical regions. Traditional heat stress metrics assume light work and unlimited thermoregulatory capacity, systematically underestimating risk for laborers. This study modifies the Extended Heat Index to incorporate realistic metabolic loads and empirical physiological limits on sweating (2 L/h) and vasodilation (7.8 L/min) based on field studies. Results show that safe working thresholds shift to lower temperature-humidity combinations than predicted by conventional indices, with critical implications for heat action plans, occupational safety standards, and climate adaptation policy in vulnerable regions.}

\authorcontributions{E.K.: Conceptualization, Methodology, Investigation, Software, Formal analysis, Data curation, Visualization, Writing – original draft. A.G.: Conceptualization, Supervision, Funding acquisition, Writing – review \& editing.}

\authordeclaration{The authors declare no competing interests.}

\correspondingauthor{\textsuperscript{1}To whom correspondence should be addressed. E-mail: elifkilic@berkeley.edu}

\keywords{heat index $|$ heat stress $|$ thermoregulation $|$ occupational health $|$ climate adaptation}

\begin{abstract}
\textbf{Importance:} Global warming is already pushing roughly 600 million people outside the human climate niche ($\sim$13 °C). Each additional degree of warming is projected to push another 10\% of the global population out of this climate niche.

\textbf{Gap:} When coupled with high humidity, the high temperatures we already experience present severely underestimated public health risks by impairing the body's ability to dissipate heat through sweating. Conventional heat stress metrics (wet-bulb temperature and the Heat Index) fail to capture the critical thresholds of occupational temperature–humidity combinations.

\textbf{Objectives:} This study modifies the Extended Heat Index (EHI) to reflect the realities of occupational heat stress affecting 70\% of the global workforce. We address: (1) How do heat stress estimates change when metabolic fluxes increase from light to heavy labor? (2) How much do traditional indices underestimate risk for workers in hot, humid environments? (3) Can a physiologically limited model better identify thermoregulatory thresholds for vulnerable populations?

\textbf{Method:} We introduce EHI-N, a modified EHI that includes (i) variable metabolic heat production representative of heavy manual work (3-6 METs, 174-348 W/m²), (ii) physiological caps on evaporative cooling through eccrine sweating rates and circulatory cooling through cutaneous vasodilation rates based on field and laboratory observations (sweat rate capped at $\sim$2 L h$^{-1}$ and skin blood flow capped at $\sim$7.8 L min$^{-1}$), and (iii) a smooth minimum function to handle sweat capacity limits while preserving monotonicity.

\textbf{Key Findings:} Increasing metabolic load and imposing physiological limits shifts the onset of uncompensated heat stress to lower temperature–humidity combinations, reducing safe working thresholds by tens of degrees Celsius across physiological regions.

\textbf{Implications:} The framework underscores the need for vulnerability-based heat indices that are scientifically rigorous, socially responsive, and operationally feasible.
\end{abstract}

\dates{This manuscript was compiled on \today}
\doi{\url{www.pnas.org/cgi/doi/10.1073/pnas.XXXXXXXXXX}}

\maketitle
\thispagestyle{firststyle}
\ifthenelse{\boolean{shortarticle}}{\ifthenelse{\boolean{singlecolumn}}{\abscontentformatted}{\abscontent}}{}

\section*{Introduction}

\subsection*{Global warming and heat extremes}

Across the world, 2023 and 2024 were the hottest years on record. Many regions experienced prolonged heat waves that exceeded both historical norms and human tolerance. The 2024 heat anomaly exemplifies the accelerating crisis: global temperatures reached 1.28°C above the 1951–1980 baseline, exceeded the 1.5°C Paris target, and simultaneously exposed 4.1 billion people to extreme heat conditions. Low-latitude regions experienced the most intense combinations of heat and humidity, yet their populations often lack adaptive infrastructure such as air conditioning.

The growing severity of extreme heat events represents one of the most urgent and inequitable consequences of anthropogenic climate change. These events are becoming more frequent and increasingly concentrated in areas where populations are least equipped to adapt. As the climate continues to warm, heat-related illnesses, productivity losses, and mortality are projected to rise sharply, especially in low- and middle-income countries and among economically marginalized communities.

Human civilization has thrived for the last 6,000 years in a narrow range of climate conditions known as the human climate niche, with mean annual temperatures near 13°C. Anthropogenic warming is now pushing large populations beyond this habitable zone. Recent studies suggest that over 600 million people already live outside the human climate niche, and with each additional degree of warming, 10\% of the global population will join them. By 2070, between 1 and 3.5 billion people could face intolerable heat and humidity conditions that may necessitate migration.

\subsection*{Human physiological limits and the human climate niche}

Prolonged extreme heat, particularly when coupled with high humidity, poses severe public health risks, including heat stroke, cardiovascular failure, chronic kidney disease, cognitive impairment, and death from hyperthermia. These risks are amplified in regions with low resources and inadequate infrastructure, and are unequally distributed across socioeconomic strata, gender, and occupational groups.

Human thermoregulation relies on a delicate balance between metabolic heat production and heat dissipation through evaporation, radiation, and convection. When dissipation mechanisms cannot match internal heat production, core body temperatures rise, leading to heat-related illness and potentially death.

The body's primary thermoregulatory mechanisms include:
\begin{itemize}
\item \textbf{Evaporative cooling through sweating}, the most effective heat dissipation mechanism in hot environments. This is constrained by water availability and physiological limits of approximately 2 liters per hour.
\item \textbf{Cutaneous vasodilation}, where blood is shunted to the skin surface to facilitate heat transfer. This is limited by maximum cardiac output ($\sim$7.8 L/min).
\end{itemize}

The critical role of humidity: high vapor pressure prevents sweat evaporation, leaving only sensible heat loss. Recent work by Raymond, Vecellio and colleagues has challenged longstanding assumptions about survivability thresholds. Their findings demonstrate that critical limits occur at wet-bulb temperatures between 20-32°C depending on activity level and individual physiology, substantially lower than the previously assumed 35°C limit. These limits are activity-dependent and physiology-dependent.

\subsection*{Current heat stress assessment approaches and gaps}

Since Steadman's 1979 work, the Heat Index (HI) has been used to communicate humid heat danger by combining air temperature and relative humidity to reflect human thermoregulation. While useful for public warnings, this model is only valid for young healthy adults performing light labor in the shade with optimal thermoregulatory systems.

Dozens of alternative indices have been proposed since, yet many inherit the HI's limitations:
\begin{itemize}
\item Most assume sedentary or light activity (low metabolic activity)
\item Many assume unlimited thermoregulatory capacity
\item Most assume steady-state and don't account for time dynamics—heat stress builds over time
\item Even the Extended Heat Index (EHI) assumes 180 W/m² (moderate activity) and unlimited capacities
\end{itemize}

However, 70\% of the global workforce performs manual labor. The Extended Heat Index (EHI) by Romps \& Lu advances the HI by modeling coupled heat and mass transfer at the skin surface, including sweat dripping at high relative humidities. This yields a more realistic stress estimate than HI within its intended range. Still, the EHI assumes near-optimal physiology and behavior, leaving out the sustained high metabolic loads typical of heavy labor.

Therefore, we develop a further extension of EHI tailored to real working conditions that are relevant to over 2.4 billion people worldwide and show how its results diverge from the EHI of Romps \& Lu.

\subsection*{India as a critical case study}

Among low-latitude countries at greatest risk, India faces acute vulnerability to humid heat extremes. Nearly half of the current workforce, over 350 million people, work outdoors often without shade. Only 10\% of households have access to air conditioning. The 2015 heat wave resulted in 2,500 deaths, and the 2022 heat wave caused widespread impacts.

India's infrastructure remains insufficient to protect against extreme weather. These conditions coincide with rapid climate change: in 2022, India's mean land surface temperature was 1°C above pre-industrial levels; by 2100, the increase is projected to reach 3.4°C under an intermediate emissions trajectory (SSP2-4.5). The frequency of heat waves in India has more than doubled in the past two decades, increasing by over 138\%.

The impacts extend beyond immediate health risks. Heat stress reduces work capacity, especially in physically demanding occupations such as agriculture, construction, street vending, transportation, and sanitation services. Documented health effects among these workers include headaches, nausea, muscle cramps, restricted lung function, reduced cardiovascular performance, musculoskeletal disorders, and in severe cases, heat stroke. For many in the informal economy, economic necessity prevents avoiding hazardous conditions, creating a feedback loop of health deterioration and reduced livelihood security.

\subsection*{Research objectives}

To address these gaps, we modify the EHI to reflect real occupational conditions by introducing three key changes:

\begin{itemize}
\item Variable metabolic heat flux to represent a range of labor intensities (3-6 METs)
\item Physiological limits on sweating and vasodilation, based on empirical human thermoregulatory capacity
\item Smooth minimum function to handle sweat capacity limits while preserving mathematical monotonicity
\end{itemize}

The overarching research questions are:
\begin{enumerate}
\item How do heat stress estimates across standard meteorological conditions change when variable metabolic heat fluxes are included and physiological limits enforced?
\item To what extent do existing indices underestimate risk for laborers performing high-exertion tasks, particularly in the tropics and subtropics?
\item Can a physiologically informed heat stress model identify critical thermoregulatory thresholds for vulnerable populations compared with traditional indices?
\end{enumerate}

By addressing these questions, this work aims to improve the accuracy and relevance of heat stress assessment tools, support evidence-based adaptation planning, and inform heat action plans, early warning systems, and public health interventions in vulnerable regions.

\section*{Background}

To understand the inadequacies of current approaches, we must first examine the physiological mechanisms that underpin human heat tolerance.

\subsection*{Physiological heat balance and thermoregulation}

Core body temperature, held close to 310 K (36.85°C), is central to human survival and performance. Core temperatures above 40°C cause protein denaturation, cellular dysfunction, and multi-organ failure, while sustained elevations above 38°C impair work capacity and judgment. Even small sustained increases can impair cognitive and physical function, and larger increases can be fatal.

This stability is maintained through a balance of heat gains (from metabolic activity, absorbed solar radiation, and environmental conduction/convection) and heat losses (primarily through convection and evaporation of sweat). The steady-state human heat balance equation is:

\begin{equation}
S = M - W_k - K - R - C - E - C_{res} - E_{res}
\end{equation}

where $S$ is heat storage, $M$ is metabolic heat production, $W_k$ is external work, $K$ is conduction, $R$ is radiation, $C$ is convection, $E$ is evaporation, $C_{res}$ is respiratory convection, and $E_{res}$ is respiratory evaporation (all in W).

When ambient temperature exceeds body temperature, evaporation becomes the only effective avenue for heat dissipation. At high humidity, which hinders evaporation, sweat drips off the skin without cooling, and the body's heat loss capacity declines.

\subsection*{Physiological limits in existing heat indices}

Two major physiological constraints are not represented in the original EHI but are relevant for modeling heat stress in India and similar contexts:

\begin{enumerate}
\item \textbf{Sweating rate limit:} EHI allows sweating rates up to 3.3 L h$^{-1}$, yet field studies in India report maxima near 2 L h$^{-1}$ and medians below 1 L h$^{-1}$. This overestimation can yield artificially low EHI values, underestimating heat stress.

\item \textbf{Skin blood flow limit:} Lu and Romps found that capping skin blood flow had minor influence on their low-activity scenarios, therefore they did not impose a cap. In contrast, experimental skin blood flow maxima are $\sim$7.8 L min$^{-1}$ under controlled conditions. Field evidence suggests real-world maxima are even lower, especially among heat-stressed agricultural workers in India. We show that at higher metabolic rates (e.g., 350 W/m²) this constraint becomes critical.
\end{enumerate}

These omissions limit the applicability of the EHI to occupational settings in hot and humid climates, where both factors are critical to heat stress risk.

To address these limitations, we modify the EHI to incorporate: (1) limits on sweating rate and vasodilation based on empirical field observations, (2) higher metabolic heat generation rates consistent with hard labor, and (3) a smooth minimum function to handle capacity limits while maintaining monotonicity. These additions are grounded in physiological literature and field measurements from heat-exposed occupational settings, ensuring that the model reflects the capacity and constraints of real-world vulnerable populations rather than idealized laboratory subjects.

\section*{Approach}

We build on the peer-reviewed EHI model based on physics of heat and mass transfer and energy balance, modifying key parameters related to internal heat generation and thermoregulatory capacity limits.

\subsection*{Model Framework}

The Extended Heat Index (EHI) solves coupled heat balance equations iteratively to find equilibrium core and skin temperatures where net heat flux equals zero. The model partitions the thermal environment into six regions based on the dominant heat transfer mechanisms:

\begin{itemize}
\item \textbf{Region I (Ventilation-limited):} Thermoregulatory control via fractional skin coverage
\item \textbf{Regions II-III (Conduction-limited):} Thermoregulatory control via clothing adjustment
\item \textbf{Region IV (Internal-limited evaporation):} Blood flow control becomes limiting
\item \textbf{Region V (Surface-saturated evaporation):} Skin is fully wet, sweat drips without cooling
\item \textbf{Region VI (Physiological limits exceeded):} Core temperature rises above tolerance limits
\end{itemize}

The flux balance equations for each region are:

\textbf{Region I:}
\begin{equation}
\Phi_1 = Q_m - Q_v(T_a, P_a, Q_m) - (1 - \phi) \frac{T_c - T_s}{R_s}
\end{equation}

\textbf{Regions II–III:}
\begin{equation}
\Phi_2 = Q_m - Q_v(T_a, P_a, Q_m) - (1 - \phi) \frac{T_c - T_s}{R_s} - \phi \frac{T_c - T_f}{R_s}
\end{equation}

\textbf{Regions IV–V:}
\begin{equation}
\Phi_3 = Q_m - \eta Q_m \left(c_{pa}(T_c - T_a) + \frac{L r_{gas,a}}{r_{gas,v} p}(P_c - P_a)\right) - 0.80 \epsilon \sigma (T_c^4 - T_a^4) - 12.3(T_c - T_a) - \frac{P_c - P_a}{Z_a}
\end{equation}

\textbf{Region VI:}
\begin{equation}
\Phi_4 = Q_m - Q_v(T_a, P_a, Q_m) - \frac{T_c - T_s}{R_s}
\end{equation}

\subsection*{Key Modifications}

\subsubsection*{1. Variable Metabolic Heat Flux}

Metabolic Equivalents of Task (MET) is commonly used to parameterize body heat generation. In this formulation, 1 MET (58.2 W/m²) represents resting heat generation. The original Heat Index (HI) and Extended Heat Index (EHI) assumed $Q_{metabolic}$ = 180 W/m², corresponding to approximately 3 METs. While appropriate for a leisurely walk on level ground, this assumption underestimates heat stress during hard labor activities.

Field studies indicate that sustained moderate to heavy labor (agricultural harvesting, road construction) can generate metabolic rates of 3-6 METs (174-348 W/m²). We allow $Q_{metabolic}$ to vary across this range, enabling assessment of heat stress under realistic working conditions.

\subsubsection*{2. Eccrine Sweating Rate Limit}

Maximum evaporative cooling is constrained by the physiological capacity for eccrine sweating. Field studies of Indian agricultural workers and laboratory measurements indicate a maximum sustainable sweat rate of approximately 2 liters per hour, corresponding to about 1000 W of latent heat loss.

However, not all sweat evaporates under all conditions. Above the evaporation threshold, excess sweat production drips from the skin without contributing to evaporative cooling. We replaced the EHI's unlimited sweating capacity with a hard upper limit of 2 L/h. When evaporative demand exceeds this threshold, the model transitions from Region V to Region VI, marking conditions where heat balance cannot be maintained.

\subsubsection*{3. Cutaneous Vasodilation Limit}

Convective heat transfer from the body's core to the skin depends on cutaneous blood flow, which increases during heat stress through peripheral vasodilation. However, cardiovascular capacity limits maximum skin blood flow to approximately 7.8 L/min, as observed in laboratory heat stress studies and consistent with field heart rate measurements during heavy labor.

This vasodilation limit was incorporated as a constraint on the conductive heat transfer term $(T_c - T_s)/R_s$ across all regional flux balance equations. When skin blood flow reaches this maximum, further increases in core temperature cannot be buffered by enhanced peripheral circulation, accelerating the rate of core temperature rise.

\subsubsection*{4. Smooth Minimum for Sweat Capacity}

A critical challenge in implementing physiological limits is maintaining mathematical monotonicity—the requirement that heat index must increase monotonically with temperature at fixed humidity. The original EHI used a hard switch at sweat capacity ($E_{actual} = \min(E_{required}, E_{capacity})$), which created discontinuities in the derivative and led to monotonicity violations.

We addressed this using a smooth minimum function based on the log-sum-exp approximation:

\begin{equation}
E_{actual} = \text{smooth\_min}(E_{required}, E_{capacity}, k) = E_{required} - \frac{1}{k} \ln(1 + e^{k(E_{required} - E_{capacity})})
\end{equation}

where $k = 0.2$ controls the transition sharpness. This approach:
\begin{itemize}
\item Guarantees monotonicity (derivative always positive)
\item Provides smooth, continuous transitions near the capacity limit
\item Captures gradual reduction in evaporative efficiency as sweat capacity is approached
\item Reduced monotonicity violations from 1,799 to 48 (97.3\% reduction) in tested conditions
\end{itemize}

The smooth minimum is physiologically justified: sweat glands don't all reach maximum simultaneously, dehydration effects build gradually, and thermoregulatory adjustments take time. While the original EHI could use explicit efficiency models (e.g., ISO 7933: $\eta = 1 - 0.5w^2$), these create negative derivatives beyond skin wettedness $w \approx 0.816$, violating monotonicity. The smooth minimum achieves similar physics phenomenologically while preserving mathematical properties required for a heat index.

\subsection*{Parameter Ranges}

To assess heat stress across diverse environmental and physiological conditions, we applied parameter ranges spanning:
\begin{itemize}
\item Air temperatures: 20-60°C
\item Relative humidity: 0-100\%
\item Human heights: 1.52-1.83 m (5-6 feet)
\item BMI: 15-30 kg/m²
\item Metabolic rates: 3-6 METs (174-348 W/m²)
\end{itemize}

For each combination of height and BMI, body mass was calculated accordingly, and body surface area was determined using the DuBois formula. Sensitivity analysis showed that EHI changes by less than 5\% across the tested height and BMI ranges at fixed metabolic rate.

\subsection*{Numerical Solution Approach}

The heat balance equations were solved iteratively using a bisection root-finding algorithm in Regions I-V and a modified Powell method in Region VI to identify equilibrium core and skin temperatures where net flux equals zero. Solver diagnostics were recorded for all simulations to monitor convergence behavior across the parameter space.

In Regions I–III (ventilation and conduction dominated regimes), the solver demonstrated robust convergence across all tested environmental conditions. However, in Regions IV-VI (evaporation limited and physiological limited regimes), occasional non-convergence occurred at air temperatures $>$30°C when the standard bracketing interval failed to contain a valid root.

This was resolved by implementing adaptive bracketing logic that dynamically expands the search interval when initial bounds prove inadequate. This modification, applied to both scalar and vectorized implementations, eliminated convergence failures within the tested environmental ranges while maintaining computational efficiency for large-scale climate data analysis.

The model was implemented in Python 3.10 using NumPy and Xarray for multi-dimensional array handling, with numerical solving accelerated using Numba just-in-time compilation and vectorization across the parameter space.

\section*{Model Validation}

\subsection*{Baseline Comparison}

To verify model fidelity under standard assumptions, we compared outputs against the original Lu \& Romps (2022) Extended Heat Index across a comprehensive temperature–humidity matrix. Under baseline conditions (metabolic rate = 180 W/m², unlimited physiological capacity), the modified model reproduced regional boundaries and equilibrium temperatures within numerical solver tolerance ($\pm$0.1°C). This validates that our implementation correctly captures the fundamental heat balance physics before applying physiological limits.

In the remainder of this paper, we use 3 METs (174 W/m²) to calculate the baseline EHI so that it is more easily mapped onto the physiological effort of leisurely walking.

\section*{Results}

\subsection*{Baseline Model Agreement}

The modified Extended Heat Index (EHI) model reproduces the outputs of the original Lu and Romps scalar implementation and the NASA JPL vectorized implementation under baseline conditions ($Q_{metabolic}$ = 180 W/m², no radiant load). Figures demonstrate complete agreement in EHI values and thermoregulatory region boundaries across the tested temperature-humidity domain, confirming the accuracy of the modified framework before applying physiological limits and variable metabolic rates.

\subsection*{Effect of Increased Metabolic Heat Flux}

Increasing $Q_{metabolic}$ from 174 W/m² (3 METs) to values representative of moderate (232 W/m², 4 METs) and heavy (290-348 W/m², 5-6 METs) exertion shifts thermoregulatory region boundaries toward lower temperature-humidity combinations. Under heavy labor conditions, Regions IV and V are triggered at temperatures below 30°C at high relative humidity. Region VI boundaries occur earlier, indicating that critical core temperature thresholds can be reached at mid-40°C ambient temperatures with RH $>$ 60\%.

[Figure: EHI comparison across MET levels showing region boundaries]

\subsection*{Impact of Limiting Thermoregulatory Capacity}

Imposing empirical upper limits on sweating (2 L/h) and vasodilation (7.8 L/min) further reduces modeled heat tolerance. Under heavy labor, maximal skin blood flow rates are reached at lower temperature-humidity combinations, and sweat-limited regions occur earlier. This increases the predicted EHI by up to 33°C in certain Regions IV and V compared to baseline assumptions.

[Figure: Difference maps showing EHI-N - EHI]

\subsection*{Physiological Region Transitions}

The classification of conditions into physiological regions reveals how thermoregulatory mechanisms shift with increasing metabolic load:
\begin{itemize}
\item At resting metabolism (1 MET), Region VI (physiological failure) only occurs at extreme temperatures ($>$50°C)
\item At moderate work (3-4 METs), Region VI begins at $\sim$45°C with high humidity
\item At heavy work (5-6 METs), Region VI occurs at $\sim$35-40°C with RH $>$ 60\%
\end{itemize}

The region boundaries shift systematically with metabolic rate, providing insight into the dominant mode of thermal strain under different work intensities.

[Figure: 1×4 layout showing region maps for MET levels 3, 4, 5, 6]

\subsection*{Core and Skin Temperature Responses}

The constrained-physiology model predicts earlier rises in core temperature ($T_{core}$) above the baseline value of 36.85°C. The safe stability zone for $T_{core}$ shrinks substantially as $Q_{metabolic}$ increases to 350 W/m². At high relative humidity, positive changes in $T_{core}$ are observed at ambient temperatures as low as 40°C, shifting Region V and VI boundaries to lower ambient temperatures.

Similarly, skin temperature ($T_{skin}$) exceeds 40°C at lower ambient temperatures compared to the original EHI. As $Q_{metabolic}$ increases to 350 W/m², skin temperatures rise above 40°C at ambient temperatures below 30°C.

\section*{Discussion}

These results reveal substantial differences in heat vulnerability that current indices overlook. We now examine what these findings mean for at-risk populations and policy.

\subsection*{Key findings in context}

This study validated a modified EHI model that explicitly incorporates variable metabolic heat flux and empirically constrained thermoregulatory limits on sweating and vasodilation. These modifications address a key shortcoming of traditional heat stress metrics: the assumption of fixed, idealized physiological capacity. By allowing $Q_{metabolic}$ to vary with labor intensity and constraining sweat and vasodilation rates to realistic values derived from field studies, the model provides a more physiologically accurate estimate of heat strain, particularly for labor-intensive occupations in heat-prone regions.

\subsection*{Implications for vulnerable populations}

The populations most affected by extreme heat often have lower maximal sweating and vasodilation capacities than the idealized laboratory subjects on which most heat stress indices are based. By incorporating empirically observed limits, this model produces heat stress estimates that more closely reflect the actual tolerance thresholds of these groups. This aligns with both scientific and ethical imperatives to ensure that climate adaptation tools are designed for those most at risk.

\subsection*{Advancing heat stress assessment}

A key advantage of the Romps and Lu EHI framework, retained in the modified version, is the classification of thermoregulatory states into physiological regions. Unlike scalar indices that output a single "feels-like" temperature, region classification reveals the mode of thermal strain (e.g., convection-limited, sweat-limited), providing insight into the mechanisms of heat stress onset. This feature enables more targeted interventions, such as hydration strategies, work–rest scheduling, and medical triage.

\subsection*{Policy Implications}

The findings from this study have direct relevance for occupational safety and public health policy in heat-prone regions. By demonstrating that traditional heat stress indices substantially underestimate risk under realistic labor conditions, the modified EHI highlights the need for policy frameworks that explicitly consider the combined effects of metabolic load and constrained thermoregulatory capacity.

In practice, this could mean:
\begin{itemize}
\item Lowering existing heat warning thresholds for work stoppages or modified duties when heavy labor is performed, particularly under high humidity
\item Integrating physiologically realistic indices into heat action plans to improve the timing and targeting of warnings
\item Updating occupational safety guidelines to reflect actual limits on sweating and vasodilation, especially for vulnerable populations with lower tolerance
\item Supporting climate adaptation investments—such as shaded work areas, hydration infrastructure, and regulated rest breaks—based on indices that account for true physiological strain rather than idealized models
\end{itemize}

Incorporating such indices into municipal early warning systems, labor codes, and public health protocols would improve protection for the millions of outdoor workers whose livelihoods and health are increasingly threatened by extreme heat.

\subsection*{Inequities in exposure and responsibility}

Extreme heat impacts are not evenly distributed. Low- and middle-income countries in Asia, Africa, Central and South America, and small island states experience the greatest number of days above health-threatening thresholds, yet high-income countries, particularly in North America and Oceania, are historically responsible for the highest per capita greenhouse gas emissions from energy (12.9–13.4 t CO₂ per person). Without significant political engagement from those most responsible, the unequal burden of climate-driven heat stress will intensify.

\subsection*{Limitations of this study}

While this study enhances the physiological realism of the EHI, several important limitations remain:
\begin{itemize}
\item This analysis excludes radiant heat flux; while deliberate in order to focus on metabolic effects, radiant load is a significant contributor to heat stress in outdoor labor contexts and will be addressed in future work
\item The applied sweating and vasodilation limits are based on available field studies, which are limited in sample size and geographic scope; broader datasets could refine these thresholds
\item The model assumes instantaneous equilibrium and does not simulate dynamic adaptation over time (e.g., sweat gland fatigue, progressive dehydration)
\item All individuals are assumed to have a baseline core temperature of 36.85°C, omitting variability in thermal tolerance
\item Skin emissivity, clothing insulation, and latent heat flux are treated as uniform across the body
\item Atmospheric pressure is held constant, excluding altitude effects
\item The model does not dynamically reduce thermoregulatory capacity based on hydration status, fatigue, or behavioral changes
\end{itemize}

\subsection*{Future Research Directions}

Future development will focus on:
\begin{itemize}
\item Integrating solar-radiant heat loads for outdoor work scenarios
\item Incorporating variable thermoregulatory efficiency to reflect differences in age, health status, and acclimatization
\item Integrating dynamic clothing insulation, hydration status, and behavioral adaptation into the model structure
\item Expanding environmental parameterization to include wind speed, surface albedo, and altitude effects
\item Developing an intelligent, geofenced early warning system that uses localized weather and physiological thresholds to send targeted alerts
\item Coupling the modified EHI with machine learning forecasts to generate short-term and seasonal occupational heat risk predictions
\item Adapting the modified EHI into a climate risk assessment platform for informing workplace safety standards, labor codes, housing regulations, and adaptive infrastructure investments
\item Collaborating with local governments, NGOs, and labor organizations to integrate physiologically realistic indices into heat action plans and occupational health policy
\end{itemize}

\section*{Conclusions}

The findings of this study carry both scientific and practical significance in the context of accelerating climate change and rising heat-health risks. By demonstrating that conventional indices such as the EHI and WBGT underestimate thermal strain when realistic metabolic heat loads and physiological constraints are considered, this research highlights critical blind spots in widely used heat stress tools.

The modified EHI model presented here—grounded in biophysical principles of human thermoregulation—provides a more physiologically realistic and adaptable framework for evaluating heat stress across diverse climatic and occupational contexts. One key implication is that public health guidance and workplace safety standards may be systematically underprotective, particularly in settings where physical exertion is high.

From a scientific perspective, the results validate the value of incorporating realistic metabolic loads into climate-health models and heat risk assessments. This provides a pathway for integrating physiological realism into reanalysis and climate projection datasets, enabling future projections of heat risk that are more meaningful for occupational health policy and adaptation planning.

Finally, the study contributes to environmental justice discourse by showing how existing indices may systematically underrepresent the risk to low-income and outdoor labor populations in the Global South. By addressing these structural blind spots, the modified EHI offers not only a technical improvement, but also a normative argument for redefining how heat risk is measured and acted upon in the 21st century.

\acknow{I thank Dr. Yi-Chuan Lu and Dr. David Romps for their mentorship on the Extended Heat Index, and Alex Goodman (NASA Jet Propulsion Laboratory) for providing the Numba-accelerated code and guidance on its implementation. I also acknowledge Berkeley Earth, Pangeo ARCO ERA5, and the European Centre for Medium-Range Weather Forecasts (ECMWF) for free access to climate datasets. This research was supported by the GEM Fellowship and the India Energy and Climate Center.}

\showacknow{}

\section*{Funding}
This research was supported by the ClimateWorks Foundation.

\section*{Declaration of Competing Interest}
The authors declare that they have no known competing financial interests or personal relationships that could have appeared to influence the work reported in this paper.

\bibliographystyle{pnas2011}

\bibliography{references}

\end{document}
